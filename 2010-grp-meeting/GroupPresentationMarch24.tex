\documentclass{beamer}
\usepackage{graphicx}
% \usepackage{beamerthemesplit} // Activate for custom appearance

\title{Eigenvalue Acceleration Methods for Denovo}
\usepackage{amsmath}
\author{Rachel Slaybaugh}
\subtitle{Neutronics Group Meeting }
\date{March 24, 2010}

\begin{document}

%-----------title slide--------------------%
\begin{frame}
  \titlepage
\end{frame}

%-------------outline slide---------------%
\begin{frame}
  \frametitle{Outline}
  
\begin{itemize}
  \item Introduction
  \item TheTransport Equation 
  \item Operator Form
  \item Krylov Methods
  \item Upscattering
  \item Eigenvalue
  \item Preconditioning
\end{itemize}

\end{frame}

%-------------slide-------------------------%
\begin{frame}
  \frametitle{Introduction}
  
\begin{itemize}
  \item My thesis topic is developing eigenvalue acceleration methods for the code Denovo.
  \item Denovo is a 3D, massively parallel, discrete ordinates transport code from ORNL.
  \item I  have added an option for using a Krylov solver to address upscattering.
  \item I will add an option to do a shifted eigenvalue solve.
  \item I will add a preconditioner.
\end{itemize}

\end{frame}

%-------------slide-------------------------%
\begin{frame}
  \frametitle{The Transport Equation}

The steady-state multigroup form of the $S_{N}$ approximation to the transport equation \cite{Den}:
\begin{align}
   \mathbf{\hat{\Omega}}_{n} \cdot \nabla \psi^{g}_{n}(\mathbf{r}) + \sigma^{g} (\mathbf{r}) \psi^{g}_{n}(\mathbf{r}) &= \sum_{g'=0}^{G}  \sum_{n'=0}^{N} \sigma_{s}^{gg'}(\mathbf{r}, \mathbf{\hat{\Omega}}_{n'} \cdot \mathbf{\hat{\Omega}}_{n}) \psi^{g'}_{n'}(\mathbf{r}) w_{n'} \nonumber \\
&+ \frac{\chi^{g}}{k} \sum_{g'=0}^{G} \nu \sigma_{f}^{g'}(\mathbf{r})\phi^{g'}(\mathbf{r}),
\end{align}
where: \\
\indent $\psi^{g}_{n} = \psi^{g}(\mathbf{\hat{\Omega}}_{n})$ is the angular flux for energy group g, \\
\indent $\sigma^{g}$ is the total group cross section, \\
\indent $\sigma^{gg'}_{s}$ is the scattering cross section from group g' to into group g, \\
\indent $\sigma^{g'}_{f}$ is the fission cross section for group g', \\
\indent $\chi^{g}$ is the fission spectrum for group g, \\
\indent $k$ is the neutron multiplication factor or eigenvalue, \\
\indent $\phi^{g} = \sum_{n=0}^{N} \psi^{g}_{n} w_{n}$ is the scalar flux for group g, and \\
\indent $w_{n}$ is a quadrature weight. \\
  
\end{frame}

%-------------slide-------------------------%
\begin{frame}
  \frametitle{Operator Form}
After discretization we can define some operators and express equation 1 as an eigenvalue problem as follows \cite{Den}:

\indent $\mathbf{L} = \mathbf{\hat{\Omega}} \cdot \nabla + \sigma$ is the transport operator, \\
\indent $\mathbf{M}$ is the operator that converts harmonic moments into discrete angles, \\
\indent $\mathbf{S}$ is the scattering matrix, \\
\indent $\mathbf{f}$ is the fission cross sections, and \\
\indent $\phi = \mathbf{D}\psi$, where $\mathbf{D}$ is the discrete-to-moment operator. 

 
\begin{equation}
  (\mathbf{I} - \mathbf{TS}) \phi = \lambda\mathbf{TF}\phi,
\end{equation}
where: \\ 
\indent $\mathbf{T} =  \mathbf{DL^{-1}M}$, \\
\indent $\mathbf{F} = \boldsymbol{\chi}\mathbf{f^{T}}$, and \\
\indent $\lambda = k^{-1}$, the eigenvalue. 

\end{frame}

%-------------slide-------------------------%
\begin{frame}
  \frametitle{Krylov Methods}
  
  \begin{itemize}
    \item A Krylov subspace of dimension $m + 1$ is $K_{m+1}  =span\{v_0, Av_0, A^2v_0,..., A^mv_0\}$ \\
    for an $n$ x $n$ matrix $A$ with starting vector $v_{0}$. \\
    \item A Galerkin method uses a discretization from an n-dimensional subspace and projects the problem of interest onto it, thereby transforming a continuous problem to a discrete problem.\\
    \item We can choose a Galerkin method onto a Krylov subspace that solves the eigenvalue problem or a problem of the form $Ax = b$ \cite{gmres}.\\
    \item We encounter both problems in Denovo - either with fission or with upscattering.
 \end{itemize}
  
\end{frame}    

%-------------slide-------------------------%
\begin{frame}
  \frametitle{Krylov Methods}
      
For non-symmetric $A$ we can use the Galerkin conditions and an orthogonal projection method to obtain the Arnoldi process \cite{gmres}:\\
    \begin {itemize}
      \item Build an orthogonal basis, $V$, for the Krylov subspace.\\
      \item Build an Upper Hessenberg matrix, $H$, where after k iterations $AV_{k} = V_{k}H_{k} + f_{k}e_{k}^{T}$.
      \item $f_{k}$ is the most recent entry added below the diagonal in $H$. 
      \item The eigenvalues of $A$ are approximated by those of $H_{k}$ and used in forming a solution,
      \item -OR- an output vector $z = V_{k}y_{k}$ can be used in solving $Ax = b$.
      \item the details of $y_{k}$ and/or how this information is used is determined by the Krylov method.
    \end{itemize}

 
\end{frame}

%-------------slide-------------------------%
\begin{frame}
  \frametitle{Upscattering with GS}
  
  \begin{itemize}
    \item Eqn. 2 gives a series of coupled single-group equations. \\
    \item When there is no upscatter then the coupled equations can be solved using forward-substitution. \\
    \item When upscatter is important in groups $[g1,G]$ then the $[g1,G]$ portion of the matrix $\mathbf{S}$ becomes semi-dense. 
    \item Denovo solves the downscattering ($[0,g1-1]$) portion using forward-sub,
    \item and the upscattering ($[g1, G]$) with Gauss-Seidel (GS) iteration, which uses the new information obtained from down and within group scattering: 
    \begin{equation}
       \mathbf{L}\psi^{k+1}_{g} = \mathbf{MS}_{gg}\phi^{k+1}_{g} + (\mathbf{M}\sum_{g'=g1}^{g-1}\mathbf{S}_{gg'}\phi^{k+1}_{g'} + \mathbf{M}\sum_{g'=g+1}^{G}\mathbf{S}_{gg'}\phi^{k}_{g'} + q_{eg}).
    \end{equation}
    \item A Krylov method is used to solve each group in series \cite{Den}.
\end{itemize}
 
\end{frame}

%-------------slide-------------------------%
\begin{frame}
  \frametitle{Upscattering with Krylov}
  
  \begin{itemize}
    \item Any purely downscattering portion of the matrix $[0, g1-1]$, is solved just as before. \\
    \item For groups $[g1,  G]$ we do \\    
    \begin{equation}
      \mathbf{L}\psi^{k+1} = \mathbf{MS}_{lft}\phi^{k} + (\mathbf{MS}_{rt}\phi^{k+1} + q_{eg}).
    \end{equation} \\
    \item where $\mathbf{S}_{rt}$ is the $[0, g1-1]$; $[g1, G]$ portion of the $\mathbf{S}$ matrix and $\mathbf{S}_{lft}$ is the $[g1, G]$; $[g1, G]$ portion of the $\mathbf{S}$ matrix.
    \item The $\mathbf{S}_{rt}$ term is already known from downscattering and is added into the source. 
    \item The resulting equation now looks like a Jacobi method instead of a Gauss Seidel method, and we can apply a Krylov solver over the entire upscatter block. 
    \item Each energy group is decoupled and we can parallelize over the upscatter energy groups. 
\end{itemize}
  
\end{frame}
 
 %-------------slide-------------------------%
\begin{frame}
  \frametitle{Eigenvalue}
  
  \begin{itemize}
    \item It is possible to add a shift to the eigenvalue problem. \\
    \begin{equation}
      (\mathbf{I} - \mathbf{T\tilde{S}}) \phi = (\lambda - \omega) \mathbf{TF}\phi \text{ ,}
    \end{equation}
    where $\mathbf{\tilde{S}} = \mathbf{S} - \omega\mathbf{F}$, and $\omega$ is the parameter by which the eigenvalue is shifted. \\
    \item This does not change the eigenvectors. 
    \item The eigenvalues near the shift become separated and easy to find. 
    \item We will use the Rayleigh Quotient Iteration method to update the shift \cite{dcs},\cite{rqi}.
    \item The $\mathbf{S}$ matrix becomes dense and we can solve the entire problem with our new Krylov block solver.
  \end{itemize}
  
\end{frame}

 %-------------slide-------------------------%
\begin{frame}
  \frametitle{Preconditioning}
  
  \begin{itemize}
    \item Based on the physical problem we know some things about the matrices of which A is comprised.\\
    \item We can use this information to create a physics-based right preconditioner: \\
    \item Defining $\tilde{q} = (\lambda - \omega) \mathbf{TF}\phi$, and multiplying Eqn. (2) by $(\mathbf{I} - \mathbf{T\tilde{S}}_{L})^{-1}$, we obtain
    \begin{equation}
      (\mathbf{I} -\mathbf{T\tilde{S}}_{L})^{-1} (\mathbf{I} -\mathbf{T\tilde{S}}) \phi = (\mathbf{I} - \mathbf{T\tilde{S}}_{L})^{-1} \tilde{q} \text{ ,}
    \end{equation}
   \item where $\mathbf{\tilde{S}} = \mathbf{\tilde{S}}_{U} + \mathbf{\tilde{S}}_{L}$.
\end{itemize}
  
\end{frame}  
  
 %-------------slide-------------------------%
\begin{frame}
 \begin{thebibliography}{4}

 \bibitem{Den} Evans, T M. \underline{Denovo Methods Manual}. Oak Ridge, TN: Oak Ridge National Laboratory, 2009. Electronic. 

 \bibitem{rqi} Parlett, B.N. ``The Rayleigh Quotient Iteration and Some Generalizations for Nonnormal Matrices.'' \emph{Mathematics of Computation} 28.127 (1974) : 679-693. Print. 
 
 \bibitem{gmres} Saad, Youcef and Martin Schultz. ``A Generalized Minimal Residual Algorithm for Solving Nonsymmetric Linear Systems.'' \emph{SIAM J. Sci. Stat. Comput.} 7.3 (1986) : 856-869. Print.
 
 \bibitem{dcs} Sorensen, Danny C. ``Implicitly Restarted Arnoldi/Lanczos Methods for Large Scale Eigenvalue Calculatoins.'' NASA Contractor Report 198342, ICASE Report No. 96-40, May 1996.

\end{thebibliography}
\end{frame}

\end{document}
