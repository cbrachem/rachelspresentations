\title{ANS Nor-Cal October Meeting}
\documentclass[12pt]{article}
\usepackage{fullpage}
\usepackage{paralist}
\begin{document}
%\maketitle

%Talk for ANS Nor-Cal meeting, October 22, 2014.

\section*{Summary}

Python for Nuclear Engineering, or PyNE (http://pyne.io/), is a collaborative, open source project consisting of a collection of
computational tools pertinent to nuclear engineering analysis and simulations.
PyNE primarily provides a common Python interface for code written in C++,
Python, and Fortran. This allows fundamental components of PyNE to easily be
combined to form powerful and complex programs. These fundamental components
include canonical nuclide and reaction naming conventions, material handling,
nuclear data and cross-section reading, mesh operations, and
physics-code-specific input and output parsing.

This presentation will begin with a discussion about the background and philosophy behind PyNE and include a demonstration of how PyNE could be used in a project. I will also cover some of the current developments, with a focus of how I'm using PyNE as a tool for my research in computational methods for neutral particle transport. 


\section*{Biography}
Dr. Slaybaugh received a BS in Nuclear Engineering from Penn State in 2006 where she served as a licensed nuclear reactor operator. Dr. Slaybaugh went on to the University of Wisconsin – Madison to earn an MS in 2008 and a PhD in 2011 in the same field, as well as a certificate in Energy Analysis and Policy. For her PhD she researched acceleration methods for massively parallel deterministic neutron transport codes. Her focus was on fixed source solvers, eigenvalue solvers, and preconditioners that could efficiently scale to hundreds of thousands of cores. Dr. Slaybaugh then worked with hybrid (deterministic-Monte Carlo) methods for shielding applications at Bettis Laboratory while teaching at the University of Pittsburgh as an adjunct faculty member. 

Throughout her career Dr. Slaybaugh has been engaged in software carpentry education and training. She is continuing and expanding these activities in her new role as an Assistant Professor of Nuclear Engineering at UC – Berkeley. At Berkeley Prof. Slaybaugh is building a research program based in computational methods and applied to existing and advanced nuclear reactors, nuclear non-proliferation and security, and shielding applications. 

\end{document}
