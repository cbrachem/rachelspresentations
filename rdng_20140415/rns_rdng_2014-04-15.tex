%%%%%%%%%%%%%%%%%%%%%%%%%%%%%%%%%%%%%%%%%%%%%%%%%%%%%%%%%%%%
%%  Class 2, NE 155
%%

\documentclass[xcolor=x11names,compress]{beamer}

\definecolor{CoolBlack}{rgb}{0.0, 0.18, 0.39}
%% General document %%%%%%%%%%%%%%%%%%%%%%%%%%%%%%%%%%
\usepackage{graphicx}
\usepackage{tikz}
\usetikzlibrary{decorations.fractals}
%%%%%%%%%%%%%%%%%%%%%%%%%%%%%%%%%%%%%%%%%%%%%%%%%%%%%%

%% Beamer Layout %%%%%%%%%%%%%%%%%%%%%%%%%%%%%%%%%%
\useoutertheme[subsection=false,shadow]{miniframes}
\useinnertheme{default}
\usefonttheme{serif}
\usepackage{palatino}
\usepackage{tabu}

% addition of color
\usepackage{xcolor}
\definecolor{dgreen}{rgb}{0.,0.6,0.}
\definecolor{RawSienna}{cmyk}{0,0.72,1,0.45}

\setbeamerfont{title like}{shape=\scshape}
\setbeamerfont{frametitle}{shape=\scshape}

\setbeamercolor*{lower separation line head}{bg=CoolBlack} 
\setbeamercolor*{normal text}{fg=black,bg=white} 
\setbeamercolor*{alerted text}{fg=red} 
\setbeamercolor*{example text}{fg=black} 
\setbeamercolor*{structure}{fg=black} 
 
\setbeamercolor*{palette tertiary}{fg=black,bg=black!10} 
\setbeamercolor*{palette quaternary}{fg=black,bg=black!10} 

\renewcommand{\(}{\begin{columns}}
\renewcommand{\)}{\end{columns}}
\newcommand{\<}[1]{\begin{column}{#1}}
\renewcommand{\>}{\end{column}}

% adding slide numbers
\addtobeamertemplate{navigation symbols}{}{%
    \usebeamerfont{footline}%
    \usebeamercolor[fg]{footline}%
    \hspace{1em}%
    \insertframenumber/\inserttotalframenumber
}

% equation stuff
\newcommand{\Macro}{\ensuremath{\Sigma}}
\newcommand{\Sn}{\ensuremath{S_N} }
\newcommand{\vOmega}{\ensuremath{\hat{\Omega}}}
\usepackage{mathrsfs}
\usepackage[mathcal]{euscript}
\usepackage{amssymb}
\usepackage{amsthm}
\usepackage{epsfig}
\usepackage{amsmath}

\newcommand{\ve}[1]{\ensuremath{\mathbf{#1}}}
\newcommand{\micro}{\ensuremath{\sigma}}
\newcommand{\detR}{\ensuremath{\Sigma}}
%%%%%%%%%%%%%%%%%%%%%%%%%%%%%%%%%%%%%%%%%%%%%%%%%%

\begin{document}

%%%%%%%%%%%%%%%%%%%%%%%%%%%%%%%%%%%%%%%%%%%%%%%%%%%%%%
%%%%%%%%%%%%%%%%%%%%%%%%%%%%%%%%%%%%%%%%%%%%%%%%%%%%%%
\begin{frame}
\title{Recent Past and Planned Research}
\subtitle{Reactor Design and Neutronics Group}
\author{
        \includegraphics[height=2cm]{bk}\\R.\ N.\ Slaybaugh}

\date{April 15, 2014}
\titlepage
\end{frame}

% --------------------------------------------------------------
\begin{frame}[fragile]{Outline}
  \frametitle{Outline}
  \begin{itemize}
    \item Hybrid methods overview
    \begin{itemize}
     	\item Background and Motivation
		\item CADIS
		\item FW-CADIS
    \end{itemize}
	\item MC importances in the presence of space and energy self-shielding
	\begin{itemize}
    		\item Background
		\item Problem Investigation
		\item Resonance Factor Method
		\item Results
		\item Summary and Conclusions
  	\end{itemize}
	\item MC importances for problems with strong anisotropies
	\item Other potential projects
  \end{itemize}

\end{frame}


% --------------------------------------------------------------
% --------------------------------------------------------------
\section{\scshape Hybrid Methods}
\subsection{Motivation}
\begin{frame}[fragile]
  \frametitle{Solving the TE}

\begin{columns}
  \begin{column}{0.5\textwidth}
  \begin{center}
  \underline{Monte Carlo}
  \end{center}
	\begin{itemize}
	\item Solution has associated statistical error
	\item Continuous phase space: ``gold standard answers"
	\item Can take a long time
	\item Good for streaming
	\item Optically thick = slow
	\end{itemize}
  \end{column}
  \begin{column}{0.5\textwidth}
  \begin{center}
  \underline{Deterministic}
  \end{center}
	\begin{itemize}
	\item Solution equally valid everywhere
	\item Discretized phase space: drives solution quality
	\item Can be fast
	\item Streaming = ray effects
	\item Good for optically thick
	\end{itemize}
  \end{column}
\end{columns}

\end{frame}

% --------------------------------------------------------------
\begin{frame}[fragile]
  \frametitle{Acceleration}
  \begin{itemize}
  	\item To use MC in many applications, we need to \textit{accelerate} it
	\item Variance reduction is designed to improve the FOM:
  \end{itemize}
\begin{align}
\text{FOM} = \frac{1}{\text{R}^2\text{t}} \qquad & \text{R = relative error} \nonumber \\ 
& \text{t = time} \nonumber 
\end{align}
  \begin{itemize}
  	\item \underline{Idea}: can we use deterministic and Monte Carlo methods together to lessen the weaknesses of each?
  \end{itemize}
  $\rightarrow$ \textbf{Hybrid Methods}

\end{frame}


% --------------------------------------------------------------
\subsection{Background}
\begin{frame}[fragile]
  \frametitle{CADIS}
Define response with function $f(\ve{r}, E)$ in volume, $V_d$ as
%
\begin{equation}
 R = \int_E \int_{V_f} f(\ve{r}, E) \phi(\ve{r}, E) dV dE 
 \label{eq:Response}
\end{equation}
\begin{columns}
  \begin{column}{0.5\textwidth}
	\begin{align}
  	H\phi &= q \quad \text{(forward)}\nonumber \\
  	%
  	H^{\dagger} \phi^{\dagger} &= q^{\dagger} \quad 
  	\text{(adjoint)}\nonumber
  	\end{align}
  \end{column}
  \begin{column}{0.5\textwidth}
  	\begin{align}
  	\langle H\phi, \phi^{\dagger} \rangle &= \langle H^{\dagger} \phi^{\dagger}, \phi \rangle \:, \text{and therefore} \nonumber \\
  	%
  	\langle q, \phi^{\dagger} \rangle &= \langle q^{\dagger}, \phi \rangle \nonumber
  	\end{align}
  \end{column}
\end{columns}
\vspace*{1 em}
If we let $q^{\dagger} = f(\ve{r}, E)$ then
%
\begin{equation}
 \langle q^{\dagger}, \phi \rangle = \langle f, \phi \rangle = R = \langle q, \phi^{\dagger} \rangle
 \label{eq:ResponseRedef}
\end{equation}
%
Eq.\ \eqref{eq:ResponseRedef} expresses that $\phi^{\dagger}$ represents the expected contribution of a source particle to the response given the source, $q$.

\end{frame}



\end{document}